\documentclass[twocolumn,hidelinks]{article}
\usepackage{orcidlink}
\usepackage[
    type={CC},
    modifier={by},
    version={4.0},
    imagedistance={.5em},
    imagewidth={7em},
    imagemodifier={-80x15},
]{doclicense}
\usepackage[margin=1in,columnsep=1em]{geometry}
\usepackage{enumitem} % to control gaps in description lists
\usepackage{url}
\usepackage[acronym]{glossaries}
\usepackage{amsthm}
\usepackage{amsmath}
\usepackage{amssymb}
\usepackage{complexity}
\usepackage{cancel}
\usepackage[ruled,vlined]{algorithm2e}
\usepackage{nicematrix}
\usepackage{listings}
\usepackage{cleveref}
\usepackage{seqsplit}
\lstset{
    basicstyle=\ttfamily,
    columns=fullflexible,
    commentstyle=\color{gray}\ttfamily,
    keepspaces=true,
}
\newcommand{\baheem}{Băhēm}
\newcommand{\alyal}{Alyal}
\newcommand\blfootnote[1]{%
  \begingroup
  \renewcommand\thefootnote{}\footnote{#1}%
  \addtocounter{footnote}{-1}%
  \endgroup
}
\DeclareMathOperator{\random}{random}
\DeclareMathOperator{\enc}{enc}
\DeclareMathOperator{\dec}{dec}
\DeclareMathOperator{\fread}{read}
\DeclareMathOperator{\fwrite}{write}
\DeclareMathOperator{\entropy}{H}
\makeglossaries
\newacronym{trng}{TRNG}{true random number generator}
\newacronym{otp}{OTP}{one-time pad}
\newacronym{xor}{XOR}{bitwise exclusive-or operation}
\newtheorem{note}{Note}
\newtheorem{lemma}{Lemma}
\newtheorem{theorem}{Theorem}
\newtheorem{tradeoff}{Trade-off}
\crefname{tradeoff}{trade-off}{trade-offs}
\renewcommand\qedsymbol{$\blacksquare$}
\newcommand{\pq}[1]{\cellcolor{red!20}#1}
\newcommand{\ph}[1]{\cellcolor{blue!20}#1}
\newcommand{\qh}[1]{\cellcolor{green!20}#1}
\hyphenation{bottle-neck}
\usepackage{cleveref} % must be loaded last
\begin{document}

\begin{center}
    \Huge
    \baheem\\
    \LARGE
    A Symmetric Cipher with Provable $128$-bit Security\\
    \normalsize
    \vspace{0.5em}
    M. Rajululkahf\,\orcidlink{0000-0001-9061-2921}\footnote{Author's
    e-mail address: \{last name\}\url{@pm.me}.  Public key: \seqsplit{
        EF91FF90DF73A9D76E4841C76D5CB15E7E909C309B307BED15BFB4
        E1183B6B9903FA78447E87F166F93B002803B99C0C72C479C253E3
        D7A5D6BDF320DC0EDBDA}.}\\
    \vspace{0.5em}
    \footnotesize
    \today
\end{center}

\section*{Overview}
This paper proposes a symmetric cipher, which I name \emph\baheem, with the
following properties:
\begin{description}
    \item[Practical:]  Requires pre-sharing only a $128$-bits key.

    \item[Provably secure:]  No cryptanalysis can degrade its security
        below $\min[\entropy(\mathbf{m}), \entropy(\mathbf{k})]$ bits of
        entropy, even under Grover's algorithm \cite{10.1145/237814.237866}
        or even if it turned out that $\P = \NP$.

    \item[Simple:]  Encryption, and decryption alike, is performed in a
        single round comprised of two additions and one \gls{xor}
        operation.  A session key is generated once with a single addition.
        Should computers cease to exist, encryption and decryption can be
        performed by hand with a pen, a paper and some fair coins, with
        relative ease.

    \item[Fast:]  Runs fast on common hardware.  Its early single-threaded
        implementation achieved similar run-time speeds to OpenSSL's
        ChaCha20 \cite{chacha20}. Faster speed is easily doable with
        parallelism and better \gls{trng} optimisations.
\end{description}

This comes at a usually-negligible cost of having an approximately
$2|\mathbf{m}|$-bit ciphertext output for a $|\mathbf{m}|$-bit cleartext
input; since space is usually not a bottleneck for most applications. 

\baheem\ is the only symmetric cipher to-date that is \emph{practical} and
\emph{provably secure}.  Other ciphers are only one of them, but not both.
For example, the \gls{otp} is provably secure but usually impractical, as
it requires pre-sharing a key that is as large as the message to encrypt.
On the other hand, state of art ciphers, such as ChaCha20
or AES \cite{aes}, are practical but not provably secure.

\blfootnote{}
\blfootnote{\vspace{-1.5em}\doclicenseThis}
\vfill
\break

\section*{Declarations}
All data used in this study is included in this paper.  The latest version
of this paper can be found
here\footnote{\url{https://codeberg.org/rajululkahf/baheem}}, and the
latest version of the implementation can be found
here\footnote{\url{https://codeberg.org/rajululkahf/alyal}}.

\section*{Notation}
\begin{description}
    \item[$\entropy(\mathbf{x})$:]  Shannon's entropy of random variable
        $\mathbf{x}$.

    \item[$\mathbf{x} + \mathbf{y} \bmod{2^{128}}$:]  Unsigned $128$-bit
        addition.

    \item[$\random(128)$:]  $128$ bits generated by a \gls{trng}.

    \item[$\mathbf{k}$:]  $128$-bit pre-shared secret key.  Must seem
        random and uniformly distributed with large enough
        $\entropy(\mathbf{k})$.  Ideally, $\mathbf{k} = \random(128)$.

    \item[$\mathbf{m}$:]  A cleartext message of $|\mathbf{m}|$ many bits.

    \item[$\lceil\frac{|\mathbf{m}|}{128}\rceil$:]  Number of $128$-bit
        blocks in cleartext $\mathbf{m}$.

    \item[$\mathbf{m}_b$:]  The $b^{\text{th}}$ $128$-bit block from
        $\mathbf{m}$.  In other words: $\mathbf{m}_0 \Vert \mathbf{m}_1
        \Vert \ldots \Vert
        \mathbf{m}_{\lceil\frac{|\mathbf{m}|}{128}\rceil} = \mathbf{m}$.

    \item[$\mathbf{s} = \random(128)$:]  Session key.

    \item[$\mathbf{p}_b = \random(128)$:]  Pad key of the $b^{\text{th}}$
        block.

    \item[$\mathbf{\hat s}, \mathbf{\hat p}_b, \mathbf{\hat m}_b$:]
        Encrypted $\mathbf{s}$, $\mathbf{p}_b$ and $\mathbf{m}_b$,
        respectively.
\end{description}

\tableofcontents

\section{Introduction}
State of art symmetric ciphers, such as ChaCha20 or AES, are attractive for
their practicality (requiring only a small, say, a $256$-bit key to
pre-share).  Their security is probable, but not provable, which is
supported by the failure of the many attempts to break them so far.

However, it remains unknown whether they are actually secure.  It is even
unknown if it is possible for a function in their class to exist, since it
remains unknown whether $\P \ne \NP$. This uncertainty about their security
is quite risky, as encrypted sensitive data is often exposed over public
networks.  Should such ciphers be discovered to be broken, the previously
encrypted data are effectively exposed.  In other words, such ciphers offer
the following trade-off:

\begin{tradeoff}[State of art]
    \emph{\underline{Enjoy}} pre-sharing in advance only a small key
    $|\mathbf{k}|=256$, and $|\mathbf{\hat m}| = |\mathbf{m}|$.
    \emph{\underline{In return}} give up provable security.
    \label{tradeoff_otp}
\end{tradeoff}

On the other hand, Shannon's \gls{otp} is more than just provably secure,
as it satisfies the higher criteria of having \emph{perfect secrecy}; that
is, no cryptanalysis can degrade its security below $\entropy(\mathbf{m})$
many bits.

However, the \gls{otp} is usually impractical as it requires the
communicating parties to exchange keys that are as large as the size of the
messages that they will be exchanging in the future.  This often implies
the necessity to exchange many gigabytes, or terabytes, of true random bits
in advance of the communication, which is too difficult to satisfy with
most application scenarios.  In other words, \gls{otp} offers the following
trade-off:

\begin{tradeoff}[\gls{otp}]
    \emph{\underline{Enjoy}} $\entropy(\mathbf{m})$-bit provable security,
    and $|\mathbf{\hat m}| = |\mathbf{m}|$.
    \emph{\underline{In return}} pre-share in advance a random pad that is
    as large as the size of the sum of all messages that you will be
    exchanging in the future.  For example, it could be that $|\mathbf{k}|
    > 8 \times 10^{12}$ (terabytes).
    \label{tradeoff_otp}
\end{tradeoff}

Due to \gls{otp}'s impractically, most applications choose to rather adopt
the \emph{practically} secure (but not provably) ciphers like ChaCha20 or
AES, in order to avoid the unscalable constraint of having to exchange
large random bits in advance of their communication.

\baheem\ offers a unique trade-off in order to save provable security,
while maintaining practicality for most applications. Specifically:

\begin{tradeoff}[\baheem]
    \emph{\underline{Enjoy}} pre-sharing in advance only a $128$-bit key,
    and a $\min[\entropy(\mathbf{m}), \entropy(\mathbf{k})]$-bit provable
    security.
    \emph{\underline{In return}} $|\mathbf{\hat m}| \approx 2|\mathbf{m}|$.
    \label{tradeoff_baheem}
\end{tradeoff}

\Cref{tradeoff_baheem} is quite interesting as it does not require
excessive planing in advance (as in the \gls{otp} case) with a compromise
that is only a polynomial increase in space, which is highly tolerable in
most real world scenarios, or even unnoticeable.

Common applications, such as instant messaging, emails, monetary
transactions, password databases, etc, often exchange small enough data
that effectively make the use of \baheem\ unnoticeable from an end user
perspective.

The tests in \cref{sec_benchmark} show that the run-time difference between
\alyal's \baheem\ and OpenSSL's ChaCha20 implementations are extremely
similar when encrypting and decrypting a $500$ megabytes file, supporting
that \baheem's space overhead is negligible in practice.


\section{Proposed Algorithm: \baheem}
\Cref{alg_enc,alg_dec} show \baheem's encryption and decryption by which
the process is repeated over every $128$-bit blocks of $\mathbf{m}$:
$\mathbf{m}_0, \mathbf{m}_1, \ldots,
\mathbf{m}_{\lceil\frac{|\mathbf{m}|}{128}\rceil}$.

The reason for choosing a $128$-bit block is only for its implementation
simplicity with common hardware.  An $|\mathbf{m}|$-bit block could be more
suitable for the pen and paper method.

Communicating peers do not have to exchange a new session key $\mathbf{s}$
with every message that they send.  It suffices them to exchange it only
with their first message, and then re-use it indefinitely.

\begin{algorithm}[tbh]
    \SetKwInOut{Input}{input}
    \SetKwInOut{Output}{output}
    \Input{$\mathbf{k}, \mathbf{m}_0, \mathbf{m}_1, \ldots$}
    \Output{$
        \mathbf{\hat s},
        (\mathbf{\hat p}_0, \mathbf{\hat m}_0),
        (\mathbf{\hat p}_1, \mathbf{\hat m}_1),
        \ldots
    $}
    \hrulefill\\
    $\mathbf{s} \gets \random(128)$\\
    $\mathbf{\hat s} \gets \mathbf{s} + \mathbf{k} \bmod{2^{128}}$\\
    \For{$b \in (0, 1, \ldots, \lceil\frac{|\mathbf{m}|}{128}\rceil-1)$}{
        $\mathbf{p}_b \gets \random(128)$\\
        $\mathbf{\hat p}_b
            \gets \mathbf{p}_b + \mathbf{k}
            \bmod{2^{128}}$\\
        $\mathbf{\hat m}_b
            \gets \mathbf{m}_b \oplus (\mathbf{p}_b + \mathbf{s}
            \bmod{2^{128}})$\\
    }
    \caption{\baheem\ encryption}
    \label{alg_enc}
\end{algorithm}

\begin{algorithm}[tbh]
    \SetKwInOut{Input}{input}
    \SetKwInOut{Output}{output}
    \Input{$
        \mathbf{k},
        \mathbf{\hat s},
        (\mathbf{\hat p}_0, \mathbf{\hat m}_0),
        (\mathbf{\hat p}_1, \mathbf{\hat m}_1),
        \ldots
    $}
    \Output{$\mathbf{m}_0, \mathbf{m}_1, \ldots$}
    \hrulefill\\
    $\mathbf{s} \gets \mathbf{\hat s} - \mathbf{k} \bmod{2^{128}}$\\
    \For{$b \in (0, 1, \ldots, \lceil\frac{|\mathbf{m}|}{128}\rceil-1)$}{
        $\mathbf{p}_b
            \gets \mathbf{\hat p}_b - \mathbf{k}
            \bmod{2^{128}}$\\
        $\mathbf{m}_b
            \gets \mathbf{\hat m}_b \oplus (\mathbf{p}_b + \mathbf{s}
            \bmod{2^{128}})$\\
    }
    \caption{\baheem\ decryption}
    \label{alg_dec}
\end{algorithm}


\section{Security Analysis}
The \baheem\ encryption is a variation of Shannon's \gls{otp}, the
\gls{xor} cryptosystem:
\[
    \mathbf{\hat m}_b \gets \mathbf{m}_b \oplus
    \underbrace{
        (\mathbf{p}_b + \mathbf{s} \bmod{2^{128}})
    }_{\text{One-time encryption pad}}
\]

It trivially follows from Shannon's perfect secrecy proof of the \gls{otp}
\cite{perfect_secrecy} that \baheem\ is secure if its encryption pad
maintains its security.

To simplify the analysis, suppose that the size of a block in \baheem\ is
$3$ bits only, and that the cleartext block $\mathbf{m}_b$ is known to the
adversary, which implies that the adversary can trivially know that:
\[
    \mathbf{p}_b + \mathbf{s} \bmod{2^3}
    = \mathbf{\hat m}_b \oplus \mathbf{m}_b
\]
in addition to adversary's knowledge of the public variables $\mathbf{\hat
s}$ and $\mathbf{\hat p}_b$. More specifically, suppose that the
adversary found that:
\begin{align*}
    0&=\mathbf{\hat s} = \mathbf{s} + \mathbf{k} \bmod{2^3}  \\
    3&=\mathbf{\hat p}_b = \mathbf{p}_b + \mathbf{k} \bmod{2^3}  \\
    5&=\mathbf{\hat m}_b \oplus \mathbf{m}_b
            = \mathbf{p}_b + \mathbf{s} \bmod{2^3}\\
\end{align*}

Then, the question is:  will this information reduce the space from which
the key $\mathbf{k}$ is chosen from?  In other words, what are the possible
values of $\mathbf{k}$ that can lead to the outputs $0$, $3$ and $5$ above?
\Cref{tbl_3bit_add} visualises this.

\begin{table}[tbh]
\centering
\begin{NiceTabular}{p{.1em}c|cccccccc}[colortbl-like]
    \        &&&&& $\mathcal{Y}$ &&&& \\
             &   &     0 &     1 &     2 &     3 &     4 &     5 &     6 &     7 \\\hline
             & 0 & \pq 0 &     1 &     2 & \qh 3 &     4 & \ph 5 &     6 &     7 \\
             & 1 &     1 &     2 & \qh 3 &     4 & \ph 5 &     6 &     7 & \pq 0 \\
             & 2 &     2 & \qh 3 &     4 & \ph 5 &     6 &     7 & \pq 0 &     1 \\
$\mathcal{X}$& 3 & \qh 3 &     4 & \ph 5 &     6 &     7 & \pq 0 &     1 &     2 \\
             & 4 &     4 & \ph 5 &     6 &     7 & \pq 0 &     1 &     2 & \qh 3 \\
             & 5 & \ph 5 &     6 &     7 & \pq 0 &     1 &     2 & \qh 3 &     4 \\
             & 6 &     6 &     7 & \pq 0 &     1 &     2 & \qh 3 &     4 & \ph 5 \\
             & 7 &     7 & \pq 0 &     1 &     2 & \qh 3 &     4 & \ph 5 &     6 \\
\end{NiceTabular}
\caption{Exhaustive unsigned $3$-bit addition.  For a given output
$\mathbf{x} + \mathbf{y} \bmod{2^3}$, there are $2^3$ many possible input
values of $(\mathbf{x}, \mathbf{y}) \in \mathcal{X}\times\mathcal{Y}$ that
map to $\mathbf{x} + \mathbf{y} \bmod{2^3}$.}
\label{tbl_3bit_add}
\end{table}

As shown in \cref{tbl_3bit_add}, the total number of horizontal, or
vertical, intersections that simultaneously cross all of the outputs $0$,
$3$ and $5$, remain $2^3$.  Meaning, the total number of values of
$\mathbf{k}$ that could lead to the outputs remains $2^3$.

This $3$-bit example can be trivially extended by induction to show that
the same conclusions hold even with a $128$-bit unsigned addition and any
other output numbers than $0$, $3$ and $5$.

Therefore, we can conclude that adversary's knowledge of the public
variables $\mathbf{\hat s}$, $\mathbf{\hat p}_b$, $\mathbf{\hat m}_b$ and
the cleartext $\mathbf{m}_b$, which leads to deducing $\mathbf{p}_b +
\mathbf{s} \bmod{2^{128}}$, can not reduce the space from which
$\mathbf{k}$, $\mathbf{s}$ and $\mathbf{p}_b$ are sampled.

If $\mathbf{k}$, $\mathbf{s}$ and $\mathbf{p}_b$ are generated by a
\gls{trng}, then any of the $2^{128}$ many possiblities are equally likely
to correspond to the actual values of $\mathbf{k}$, $\mathbf{s}$ and
$\mathbf{p}_b$.  In other words:
\[
    \entropy(\mathbf{k},\mathbf{s},\mathbf{p}_b
        | \mathbf{\hat s},
          \mathbf{\hat p}_b,
          \mathbf{\hat m}_b,
          \mathbf{m}_b
    )
    = 128
\]

However, since $\mathbf{k}$ could be derived from a password, such that it
looks random, but with an entropy $\entropy(\mathbf{k}) \le 128$, and since
finding any of the numbers $\mathbf{k}$, $\mathbf{s}$ and $\mathbf{p}_b$
deterministically leads to finding the others, therefore it follows that:
\[
    \entropy(\mathbf{k},\mathbf{s},\mathbf{p}_b
        | \mathbf{\hat s},
          \mathbf{\hat p}_b,
          \mathbf{\hat m}_b,
          \mathbf{m}_b
    )
    = \entropy(\mathbf{k})
\]

The numbers $\mathbf{s}$ and $\mathbf{p}_b$ are generated by a \gls{trng}
by definition, therefore the weakest element in the chain can only be
$\mathbf{k}$.

Since the public variables $\mathbf{\hat s}$, $\mathbf{\hat p}_b$ and
$\mathbf{\hat m}_b$, and the cleartext $\mathbf{m}_b$ are exhaustively all
of the outputs of \baheem\ that can be accessible to an adversary, and
since they can not reduce \baheem's private variables' space below
$\entropy(\mathbf{k})$, therefore no cryptanalysis can reduce their entropy
below $\entropy(\mathbf{k})$.

\begin{lemma}[Secure private values]
    \[
        \entropy(\mathbf{k},\mathbf{s},\mathbf{p}_b
            | \mathbf{\hat s},
              \mathbf{\hat p}_b,
              \mathbf{\hat m}_b,
              \mathbf{m}_b
        )
        = \entropy(\mathbf{k})
    \]
    \label{thrm_baheem_secure_private_values}
\end{lemma}

It is trivially implied from \cref{thrm_baheem_secure_private_values} that,
since the private values $\mathbf{s}$ and $\mathbf{p}_b$ maintain an
entropy of $\entropy(\mathbf{k})$, so does their $128$-bit summation
$\mathbf{s} + \mathbf{p}_b \bmod{2^{128}}$, which is \baheem's \gls{xor}
encryption pad.  Therefore, \baheem's encryption pad has to be secure as
well.

\begin{lemma}[Secure encryption pad]
    \[
        \entropy(
            \mathbf{s} + \mathbf{p}_b \bmod{2^{128}}
            | \mathbf{\hat s},
              \mathbf{\hat p}_b,
              \mathbf{\hat m}_b
        )
        = \min[\entropy(\mathbf{m}_b), \entropy(\mathbf{k})]
    \]
    \label{thrm_baheem_secure_encryption_pad}
\end{lemma}

Since \baheem\ is an \gls{xor} cryptosystem, and since its encryption pad
is $\entropy(\mathbf{k})$-bits secure
(\cref{thrm_baheem_secure_encryption_pad}), therefore it necessarily
follows by Shannon's perfect secrecy \cite{perfect_secrecy} that \baheem's
encryption is either $\entropy(\mathbf{k})$-bits secure, or
$\entropy(\mathbf{m}_b)$-bits secure, whichever is smaller.

\begin{theorem}[Secure encryption]
    \[
        \entropy(
            \mathbf{m}_b
            | \mathbf{\hat s},
              \mathbf{\hat p}_b,
              \mathbf{\hat m}_b
        )
        = \min[\entropy(\mathbf{m}_b), \entropy(\mathbf{k})]
    \]
    \label{thrm_baheem_secure}
\end{theorem}


\section{Benchmark}\label{sec_benchmark}
This is a benchmark that was performed on a computer with a 3.4GHz Intel
Core i5-3570K CPU, 32GB RAM, 7200 RPM hard disks, Linux
5.17.4-gentoo-x86-64, OpenSSL 1.1.1n and Alyal v3.

\begin{table}[tbh]
    \centering
    \begin{tabular}{rcll}
                & OpenSSL            & \multicolumn{2}{c}{\alyal}                \\
                & ChaCha20           & \multicolumn{2}{c}{\baheem}               \\
                &                    & \texttt{/dev/random} & \texttt{file.rand} \\\hline
        Encrypt & \textbf{0.90} secs & 2.58 secs            & 1.38 secs          \\
        500MB   & \textbf{1.06} secs & 2.60 secs            & 1.35 secs          \\
                & \textbf{1.04} secs & 2.58 secs            & 1.35 secs          \\\hline
        Decrypt & 0.89 secs          & \multicolumn{2}{c}{\textbf{0.82} secs}    \\
        500MB   & 1.12 secs          & \multicolumn{2}{c}{\textbf{0.87} secs}    \\
                & 1.06 secs          & \multicolumn{2}{c}{\textbf{0.82} secs}    \\
    \end{tabular}
    \caption{Wall-clock run-time comparison between OpenSSL's ChaCha20, and
    \alyal's \baheem\ implementation with two sources as the \gls{trng}:
    \texttt{/dev/random} and \texttt{file.rand};  the latter is simply
    \texttt{/dev/random} that was prepared in advance.}
    \label{tbl_benchmark}
\end{table}

\Cref{tbl_benchmark} shows that, while the early \baheem\ prototype,
\alyal, has a faster decryption run-time than OpenSSL's ChaCha20, it has
a slower encryption run-time. However:
\begin{enumerate}
    \item The differences in run-time are insignificant for most
        applications, which proves \baheem's practical utility in the real
        world.
    \item \baheem's provable security should arguably justify waiting the
        extra seconds, or fractions of seconds in case the \gls{trng} is
        prepared in advance, for the 500MB data, specially that many user
        applications involve encrypting much smaller data sizes with
        unnoticeable time difference
    \item Preparing the random bits in advance significantly reduces the
        encryption time as shown with the \texttt{file.rand} case in
        \cref{tbl_benchmark}, and can be optimised further should it be
        prepared in memory.
    \item \alyal\ is currently single-threaded despite \baheem's capacity
        for high parallelism as all blocks are independent. This gives room
        for future versions to be significantly faster.
\end{enumerate}

\section{Conclusions}

This paper described \baheem; a provably-secure, yet practical, symmetric
cipher.  \baheem is variation of Shannon's \gls{otp}, where the one-time
pad is securely derived from a $128$-bit pre-shared key, in a way to solve
\gls{otp}'s key impracticality of requiring a pre-shared key that is as
large as the sum of all message to encrypt in the future.  The trade-off of
\baheem\ is practically quite negligible as confirmed by benchmarks
presented in this paper, which is that the ciphertext is approximately
twice as large as the cleartext.

\bibliographystyle{unsrt}
\bibliography{references}

\vfill
\break

\appendix
\section{Implementation Examples}
\subsection{C Functions}
\Cref{lst_baheem_session_enc,lst_baheem_session_dec} show example C
functions for encrypting and decrypting session keys.

\Cref{lst_baheem_block_enc,lst_baheem_block_dec} show the same but for
encrypting and decrypting cleartext and ciphertext blocks, respectively.

In these examples, all encryptions and decryptions happen in-place whenever
possible, so the caller does not have to allocate separate memory for the
output.  The only excepton is \cref{lst_baheem_session_enc}, where the
unencrypted session key is required to encrypt the subsequent cleartext
blocks. Also, since $128$-bit wide CPU instructions are not common, the
examples operate in $64$-bit basis, each time with a different $64$-bit
part of the pre-shared and session keys.

\begin{lstlisting}[language=C, caption=Session key encryption function
                   example., label=lst_baheem_session_enc]
void baheem_session_enc(
    uint64_t *k,    /* pre-shared key */
    uint64_t *s,    /* session key    */
    uint64_t *s_enc /* encrypted s    */
) {
    s_enc[0] = s[0] + k[0];
    s_enc[1] = s[1] + k[1];
}
\end{lstlisting}

\begin{lstlisting}[language=C, caption=Session key decryption function
                   example., label=lst_baheem_session_dec]
void baheem_session_dec(
    uint64_t *k, /* pre-shared key */
    uint64_t *s  /* session key    */
) {
    s[0] -= k[0];
    s[1] -= k[1];
}
\end{lstlisting}

\begin{lstlisting}[language=C, caption=Block encryption function example.,
                   label=lst_baheem_block_enc]
void baheem_block_enc(
    uint64_t *k, /* pre-shared key    */
    uint64_t *s, /* session key       */
    uint64_t *p, /* pad keys          */
    uint64_t *m, /* message           */
    size_t  len  /* length of m and p */
) {
    size_t i;
    for (i = 0; i < len; i += 2) {
        m[i]   ^= p[i]   + s[0];
        m[i+1] ^= p[i+1] + s[1];
        p[i]   += k[0];
        p[i+1] += k[1];
    }
}
\end{lstlisting}

\begin{lstlisting}[language=C, caption=Block decryption function example.,
                   label=lst_baheem_block_dec]
void baheem_block_dec(
    uint64_t *k, /* pre-shared key    */
    uint64_t *s, /* session key       */
    uint64_t *p, /* pad keys          */
    uint64_t *m, /* message           */
    size_t  len  /* length of m and p */
) {
    size_t i;
    for (i = 0; i < len; i += 2) {
        p[i]   -= k[0];
        p[i+1] -= k[1];
        m[i]   ^= p[i]   + s[0];
        m[i+1] ^= p[i+1] + s[1];
    }
}
\end{lstlisting}

\subsection{A File Encryption Tool: Alyal}
\alyal\ is an single-threaded implementation to demonstrate \baheem's
practical utility with real-world scenarios.  Internally, Alyal uses the
functions in
\cref{lst_baheem_session_enc,lst_baheem_session_dec,lst_baheem_block_enc,lst_baheem_block_dec}.

\subsubsection{Installation}
\begin{verbatim}
git clone \
  https://codeberg.org/rajululkahf/alyal
cd alyal
make
make test
\end{verbatim}

\subsubsection{Usage}

\begin{verbatim}
alyal (enc|dec) IN OUT [TRNG]
alyal help
\end{verbatim}

To encrypt a cleartext file \texttt{a} and save it as file \texttt{b}:
\begin{verbatim}
alyal enc a b
\end{verbatim}

To decrypt the latter back to its cleartext form and save it as file
\texttt{c}:
\begin{verbatim}
alyal dec b c
\end{verbatim}

\end{document}
